\begin{abstract}
\section*{Abstract}
    \par In diesem Dokument werden die Ergebnisse der Bearbeitung des ersten Kurses des Workshops Elektro-
    und Informationstechnik wiedergeben und diskutiert. Hierbei handelt es sich um die Energiegewinnung
    mittels einer Solarzelle und der Zwischenspeicherung dieser Energie.
    Ziel dieser Arbeit ist die Untersuchung und Vermessung einer Solarzelle. Als erstes wird dafür die I-U-
    Kennlinie und die daraus ableitbare Leistung bestimmt. Danach wird an die Schaltung ein Kondensator
    angeschlossen. Sein Verhalten wird beim Laden sowie Entladen beobachtet und analysiert. Zuletzt wird
    das Verhalten des gesamten Aufbaus betrachtet.
    \par Zur Ausführung dieses Projekts werden eine Solarzelle, verschiedene Lichtquellen , das TI-Board als A/D-
    Wandler und Messgerät, ein Speicherkondensator sowie weitere elektronische Komponenten (z.B.
    Wiederstände und Dioden) verwendet.
    \par Zum Einarbeiten in die Thematik und um die Funktionsweise einiger Bauteile nachzuvollziehen, wurden Literaturen herangezogen. Für das Arbeiten mit den Messwerten wurde darüber hinaus die Software MATLAB des Unternehmens MathWorks eingesetzt. 
    \par Grundsätzlich kommt man zur Schlussfolgerung, dass die Leistungen in Abhängigkeit von der
    Beleuchtung variieren, dass die gemessenen Spannungen der Solarzelle unter den natürlichen
    Bedingungen von vielen Faktoren abhängen und abschließend dass Kondensatoren als Energiespeicher
    dienen, jedoch nicht optimal sind. Anschließend werden der Verlauf und die Resultate des Workshops in Latex dokumentiert.
\end{abstract}

