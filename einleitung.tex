\par Nachhaltigkeit und Zukunftsfähigkeit bei der Energiegewinnung nehmen heutzutage an Bedeutung immer mehr zu. Hierbei spielt besonders regenerative Energieerzeugung eine immer größer werdende Rolle, vor allem auch hinsichtlich der Umweltbelastung herkömmlicher Energieerzeugungsmethoden. Unter regenerative Energieerzeugung fallen unter anderem Windenergie, Wasserkraft, Geothermie und Photovoltaik, welche alle auf die Solarstrahlung basieren. Dabei hat in den letzen Jahren die Bedeutung der Photovoltaik, die als einzige direkt die Solarenergie in elektrische Energie umwandelt, immer weiter zugenommen.
\par Erneuerbare Energiequellen sind aufgrund vieler Aspekte attraktiv. Bereits der Name impliziert, dass deren Energien nahezu unbegrenzt zur Verfügung stehen. Für die Nachhaltigkeit ist auch gesorgt, da die Energieerzeugung ohne Emissionen erfolgt und damit weit weniger umweltbelastend ist als herkömmliche Methoden. Es entstehen keine Kosten für jegliche Brennstoffe, im Gegensatz z.B zur Kohleenergie. Dazu kommt, dass die Energiequelle grundsätzlich vom Standort unabhängig ist, weshalb auch Entwicklungsländer in der Lage sind, ihre eigene Energie zu produzieren.
\par Allerdings handelt es sich bei regenerativer Energieerzeugung noch um eine relativ neue Technologie, weshalb sie wirtschaftlich in vielen Faktoren mit herkömmlichen Energieerzeugungsverfahren nicht mithalten kann. So ist der Ertrag im Gegensatz z.B zu dem von Kohle sehr gering, weshalb mit großen Investitionen gerechnet werden muss, wenn herkömmliche durch regenerative Quellen ersetzt werden sollen. Die hohen Kosten, die dadurch entstehen, machen sie wirtschaftlich unattraktiv. Da die regenerative Erzeugung von Energie stark vom Umfeld der Anlagen hinsichtlich der Umweltfaktoren abhängt, die diese produzieren, kommen weitere Kosten auf, um die Bevölkerung zuverlässig mit Energie versorgen zu können.
\par Im Vergleich zur Windenergie, Wasserkraft oder Geothermie, die indirekt durch Solarenergie angetrieben werden, wandelt Photovoltaik Solarenergie direkt in elektrische Energie um. Hierzu nutzt sie den photovoltaischen und den photoelektrischen Effekt: Solarstrahlung trifft auf die Oberfläche einer Solarzelle. Dadurch werden manche Elektronen der äußeren Schichten der Atome der Solarzelle angeregt und fließen vom Valenzband ins sogenannte Leitungsband. Eine spezielle Bearbeitung des Materials verhindert, dass die Elektronen das Leitungsband wieder verlassen. Dadurch entsteht auf der einen Seite der Solarzelle ein Elektronenüberschuss und auf der anderen Seite ein Mangel, was anschließend zu einem Strom führt.
\par Die I-U-Kennlinie beschreibt die Stromstärke in Abhängigkeit der Spannung. So gibt der Flächeninhalt dieser die Leistung wieder. Ist die Fläche unter dem Graphen maximal, handelt es sich um den Maximum Power Point: Dem Punkt, an dem die Solarzelle mit maximaler Leistung arbeitet.
\par Eine Möglichkeit zur Energiespeicherung ist es, eine Energieart in eine andere umzuwandeln. Mehrmalige Umwandlungen sind zwar mit Energieverlust verbunden, haben im Vergleich zur direkten Energiespeicherung, z.B durch Kondensatoren, verschiedene Vorteile. Allgemein betrachtet hat jede Form der Energiespeicherung ihre Stärken und Schwächen, weshalb je nach Situation andere Methoden benutzt werden müssen. Durch weitere Forschung in diesem Gebiet, kann jedoch die Effizienz vieler dieser Methoden verbessert werden.

% Frühere Version

%\par Nachhaltigkeit und Zukunftsfähigkeit bei der Energiegewinnung nehmen heutzutage an Bedeutung
%zu. Hierbei spielen besonders regenerative Energieerzeugung eine immer größer werdende Rolle
%hinsichtlich des Umweltsaspekts. Unter regenerativer Energieerzeugung fallen unter anderem
%Windenergie, Wasserkraft, Geothermie und Photovoltaik, welche alle auf die Solarstrahlung
%basieren. (vgl. S.31 Photovoltaik). Dabei hat in den letzten zehn Jahren die Bedeutung der
%Photovoltaik, welche die Solarstrahlung direkt in elektrische Energie umwandelt, immer stärker
%zugenommen (vgl. S. 7, Erneuerbare Energien in Deutschland
%$https://www.umweltbundesamt.de/sites/default/files/medien/376/publikationen/180315_uba_hg_
%eeinzahlen_2018_bf.pdf)$
%\par Erneuerbare Energien wirken aufgrund vieler Aspekte attraktiv. So impliziert bereits der Name, dass
%die Energiequellen nahezu unbegrenzt zu Verfügung stehen. Für eine nachhaltige Zukunft ist es
%ebenso vorteilhaft, dass dessen Energieerzeugung ohne Emissionen erfolgt. Infolgedessen tauchen
%keine schwerwiegenden Konsequenzen und Bedrohungen für die Umwelt auf. Im Gegensatz zu
%Kohleenergie entstehen keine Kosten für jegliche Brennstoffe. Dazu kommt, dass die Energiequelle
%nicht komplett abhängig vom Standort ist und somit überall nutzbar ist, weshalb auch
%Entwicklungsländer in der Lage sind, selber Energie zu produzieren.
%\par Allerdings handelt es sich bei regenerativer Energie noch um eine neue Art der Energieerzeugung,
%sodass sie wirtschaftlich in vielerlei Dingen nicht effektiv ist im Vergleich zu der bisherigen
%herkömmlichen Energieerzeugung. So ist die Energiedichte im Gegensatz zu der von beispielsweise
%Kohle sehr gering, weshalb mit Quantität entgegengewirkt werden muss. Dieser hohe
%Materialverbrauch ist ein weiterer negativer Aspekt. Zudem entstehen somit hohe Kosten, wenn
%man in erneuerbare Energien investieren will, was diese wirtschaftlich unattraktiv macht. So muss
%man auch das Argument der Unerschöpflichkeit relativieren, da das Energieangebot stark variiert. So
%ist der Energieertrag insbesondere bei Photovoltaik und Windkraft von Umweltfaktoren abhängig.
%Aufgrund dessen kann der Energiebedarf der Bevölkerung nur zuverlässig abgedeckt werden, indem
%die Anzahl der Kraftwerke erhöht wird, was wiederum mit hohen Kosten verbunden ist.
%Photovoltaik wandelt elektrische Energie direkt um, indem die den photovoltaischen Effekt nutzt.
%Diese lässt sich an einem Halbleiter erläutern. Lichtwellen treffen auf die eine Oberfläche und geben
%ihre Energie an Elektronen ab, welche aus ihrer Verbindung sich lösen und aufgrund der Dotierung in
%Richtung des Pluspols wandern. Auf der anderen Seite des Kabels kommen Elektronen nach, welche
%die Löcher füllen. Die ladungsfreie Zone wird überbrückt.
%\par Die I-U-Kennlinie beschreibt grafisch die Stromstärke in Abhängigkeit der Spannung. So ergibt der
%Flächeninhalt dieser die Leistung wieder. Ist diese Fläche unter dem Graphen maximal, handelt es
%sich um den Maximum Power Point, welche den Punkt markiert, an der die Solarzelle mit maximaler
%Leistung arbeitet.
%\par Eine Möglichkeit zu Speicherungen ist es, diese Energien in andere Energiearten umzuwandeln. Diese
%mehrmaligen Umwandlungen sind zwar mit Energieverlust verbunden, jedoch sind sie immer noch
%effizienter als die direkte Energiespeicherung mit Speicherkondensatoren. Allgemein lässt sich sagen,
%dass die Speichermöglichkeiten alle ihre Schwächen haben und an ihnen noch weitergeforscht
%vwerden muss, damit die Effizienz gesteigert wird.
