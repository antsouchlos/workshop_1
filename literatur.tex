\begin{thebibliography}{99}

	\bibitem{deutschland} Umweltbundesamt. {\itshape Erneuerbare Energien in Deutschland}. \url{https://www.umweltbundesamt.de/sites/default/files/medien/376/publikationen/180315_uba_hg_eeinzahlen_2018_bf.pdf}, Abrufdatum: 11. Dezember 2018.	%zu 1.2 Einleitung
	\bibitem{photovoltaik} K. Mertens. {\itshape Photovoltaik}. Fachbuchverlag Leipzig im Carl Hanser Verlag, Leipzig, 4.Auflage, 2018. %Photovoltaik
	\bibitem{kondensator} L. Constantinescu-Simon. {\itshape 	
	Handbuch elektrische Energietechnik}. Verlag Vieweg, Braunschweig, 2.Auflage, 1997. %Kondensator Grundlagen
	\bibitem{direktUmwandl} P. Gruss und F. Schüth (Hrsg.). {\itshape Die Zukunft der Energie}. Verlag Beck, München, 3. Auflage, 2008.	%direkte Umwandlung
	\bibitem{basicSolar} H-G. Wagemann, H. Eschrich. {\itshape Photovoltaik}. Verlag Vieweg + Teubner, Wiesbaden, 2. Auflage, 2010. %Grundlagen Solarzelle
	\bibitem{skript} O. Dössel. {\itshape Vorlesungsskript zur Vorlesung Lineare Elektrische Netze}. Institut für Biomedizinische Technik, 2018.
	\bibitem{verschattung} F. Konrad. {\itshape Planung von Photovoltaik-Anlagen}. Verlag Vieweg + Teubner, Wiesbaden, 2. Auflage, 2008. %Winter-Sommer
	\bibitem{diplom} R. N. H. Vasquez. {\itshape Entwicklung und Aufbau eines Solarwechselrichters mit MPP-Tracking}. \url{http://edoc.sub.uni-hamburg.de/haw/volltexte/2010/1055/pdf/Diplomarbeit_Ruediger_Heidenreich_2010.pdf}, Abrufdatum: 10. Dezember 2018. %Diplomarbeit
	\bibitem{vorteile} Baden-Württemberg/Innovationsbeirat. {\itshape Zukunft der Energieversorgung}. Verlag Springer, Berlin, 7. Auflage, 2003. %Solar Vorteile
	\bibitem{wirkungsgrad} M. Gommeringer, A. Schmitt, F. Kammerer, M. Braun. An Ultra-Efficient Maximum Power Point Tracking Circuit for Photovoltaic Inverters In: {\itshape IECON 2015 – 41st Annual Conference of the IEEE Industrial Electronics Society }. \url{https://publikationen.bibliothek.kit.edu/1000052985}, Abrufdatum: 13. Dezember 2018 %heutiger Wirkungsgrad

	
\end{thebibliography}
