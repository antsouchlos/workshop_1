\par Ziel dieses Workshops war es, erstmals Praxiserfahrungen mit einem Projekt zu sammeln. Thematisch geht es um die Energiegewinnung, respektive um Photovoltaik. Zunächst wurde sich generell mit dem Thema Energiegewinnung und der Funktionsweise einer Solarzelle befasst. Hierzu sollte selbstständig Messungen durchgeführt werden, welche im Anschluss analysiert und ausgewertet wurden, sodass man folgende Rückschlüsse hinsichtlich der optimalen Leistung sowie der Energiespeicherung ziehen konnte. Für eine möglichst effiziente Energiegewinnung wird der Maximum-Power-Point ermittelt, an der die Leistung am größten ist. Dieser ist von der verwendeten Bestrahlungsart abhängig. Beim Einsatz im Freien fallen mehrere Einflussfaktoren zusammen aufgrund der sich verändernden Witterung. So entstehen Schwankungen einerseits in kleinen Intervallen von Tagen als auch größer betrachtet in einem längeren Zeitraum aufgrund der Jahreszeiten. Infolgedessen ist die Versorgungssicherheit nicht gewährleistet, sodass nun die Energiespeicherung einer Solarzelle genauer betrachtet wurde. Nach einer theoretischen Betrachtung wurden eigene Versuche zur Kondensatorauf- bzw. entladung durchgeführt. Als Folgerung lässt sich die Solarzelle vielmehr als Stromquelle beschreiben. Der Wirkungsgrad wurde berechnet, wobei dieser von dem Erwartungswert abweicht, welches auf eine ungenaue Messung schließt. Anschließend wurde beobachtet, inwiefern der verwendete Speicherkondensator als Energiespeicher infrage kommt. So kommen wir zu dem Resultat, dass Speicherkondensatoren für Langzeiteinsätze weniger geeignet sind sowohl von der Effizienz als auch vom wirtschaftlichen Standpunkt her.

\par Zusammenfassend lässt sich sagen, dass Photovoltaik als Mittel zur Energieerzeugung großes
Potential besitzt. Jedoch ist es als zukünftige konkurrenzfähige Alternative zu herkömmliche
Methoden vonnöten, dass für eine Effizienzsteigerung in die Weiterforschung investiert wird.
